\documentclass[12pt, french]{article}
\usepackage[utf8]{inputenc}
\usepackage{babel}
\usepackage[margin=1in]{geometry}

\begin{document}

\begin{center}
\textbf{Management de la relation individuelle}\\
Alexander Hoffmann\\
\end{center}

A specific ideology influences every corporate culture which means that there are very few universal truths about successful management. Nonetheless, there are some "golden" laws that allow people to perform their duties more efficiently in managerial and leadership positions. Although individual management's effectiveness ultimately depends on a caring, optimistic, productive state of mind and a listening attitude, this is not enough. The manager needs to learn strategies that allow him to achieve the desired goal. This includes actively engaging in process improvement, collaborating with team members, learning how to process employees' feedback effectively, and gathering the team to implement project enhancements. In this paper, we will establish some best practices in individual management.

First, communication is key. Leaders need to make sure team members understand their responsibilities and have a strong understanding of how their organization functions. These communication skills include effectively identifying team members and tasks to be accomplished, using effective end-to-end communication, and ensuring that everyone understands their role and is working as a team member. As the distinction between professional and personal life is increasingly blurred, managers and their employees are gradually experiencing profound stress situations, the cause of which is not necessarily correlated with the professional background. These situations will lead to emotional stress resulting in externalization of stress and, therefore, a biased assessment of the ability to cope in the workplace. Thus emotional support could be employed for the purpose of addressing the work-related stress.

Human beings need recognition from their own people. In fact, we must work together to recognize, support, and fulfill the needs of each other. People need space to explore passions and passions to explore challenges. And, of course, they need to be protected from the negative attitudes and reactions that face individuals who choose to pursue changes that are not compatible with prevailing conceptions. Publicly rewarding employees is a good practice for a manager. Not only does this provide a incentive for top-tier employees to perform quality work, but it gives success to smaller teams that may not be able to afford to do the same.

Encouraging people to speak out openly is a good managerial practice. The pooling of thoughts will help address current and future problems and will deter demotivation because employees feel their views matter. Employees are encouraged to consider the impact of their comments and actions on other team members and colleagues. Once the decision to speak out is made, a manager must promote those who want to be a part of the decision, such as co-workers, and actively support their choice by providing constructive feedback and encouragement

The practices and best management theory scenarios covered in this paper were first developed ages ago. The theories described in this article can help companies follow business traditions and develop their own ways to further their goals. Not only have they helped increase efficiency but they have also increased service quality. Such good practices help create collaborative work cultures in which managers and employers work hand-in-hand in order to achieve a bigger goal. So, if you want to improve the health and well-being of your employees and improve their overall performance, study some of these management theories and be inspired to make a change for the better.

\end{document}
