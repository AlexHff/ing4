\documentclass[12pt]{extarticle}
\usepackage[utf8]{inputenc}
\usepackage{graphicx}
\usepackage{float}

% Disable indentation
\setlength{\parindent}{0pt}

\title{Lab 7: Network virtualization with Virtualbox}
\author{Alexander Hoffmann}
\date{\today}

\begin{document}

\maketitle

\section{NAT mode}
\textbf{1.} The IP configuration of the host machine can be determined using \texttt{ifconfig}.\\~\\
\includegraphics[scale=0.6]{resources/1-1.png}\\

\textbf{2.} We use the same command in the virtual machine.\\~\\
\includegraphics[scale=0.7]{resources/1-2.png}\\

\textbf{3.} It seems like the host machine and the virtual machine are not on the same network. This is because the VM is connected through NAT.\\

\textbf{4.} To get the DHCP server address, we use the following command:
\begin{verbatim}
sudo grep -R "DHCPOFFER" /var/log/*
\end{verbatim}
\includegraphics[scale=0.7]{resources/1-4.png}\\
This corresponds to the DHCP server address on the host machine. Now let's see which IP the DHCP server has on the VM.\\~\\
\includegraphics[scale=0.8]{resources/1-4-2.png}\\

\textbf{5.} The IP address of the NAT device is 10.0.2.15.

\textbf{6.} Since the VM is a server, it does not have a virtual interface. Therefore, we will be using \texttt{tcpdump} to capture traffic. More specifically, to filter the DHCP protocol, we use the following:
\begin{verbatim}
tcpdump -i eth0 -pvn port 67 and port 68
\end{verbatim}
Now we have to renew the DHCP lease. To do this, use:
\begin{verbatim}
dhclient enp0s3
\end{verbatim}
Which will display the following packets.\\~\\
\includegraphics[scale=0.5]{resources/1-5.png}\\
We can observe that the VM broadcasts a DHCP Request. The DHCP server then sends a DHCP Reply with a lease.

\textbf{7.} There is no direct traffic between the DHCP server and the VM. In fact, the host machine is the DHCP server for the VM. This is why we are not capturing any traffic going to the VM.\\~\\
\includegraphics[scale=0.6]{resources/1-7.png}\\

\textbf{8.} Once again, since the VM does not have a graphical interface, we will use the \texttt{ping} command to observe the traffic. Suppose we ping google.com from the VM. Here is the traffic captured by Wireshark on the host machine.\\~\\
\includegraphics[scale=0.6]{resources/1-8.png}\\
Now let's observe the ping from the host.\\~\\
\includegraphics[scale=0.6]{resources/1-8-1.png}\\
The packets are simingly the same except that the header is slightly different.\\

\section{Host-only mode}
\textbf{9.} The IP address of the host has not changed. See question 1.\\ 

\textbf{10.} Now let's take a look at the IP configuration of the VM.\\~\\
\includegraphics[scale=0.7]{resources/2-9.png}\\
The IP address of the network is 192.168.56.1/24. It is a private network linked to a virtual interface created by VirtualBox.\\

\textbf{11.} To find out the IP address of the DHCP server, we use the same command as before. This time, the DHCP server is 192.168.56.2. Note that the IP address of the interface on the host machine is 192.168.56.1.\\~\\
\includegraphics[scale=0.6]{resources/q11.png}\\

\section{Bridged mode}
\textbf{13.} For this section, we have changed location. Therefore, the IP address is not the same as previously.\\~\\
\includegraphics[scale=0.6]{resources/13.png}\\

\textbf{14.} Now that we are in bridged mode, the VM is directly connected to the same network as the host.\\~\\
\includegraphics[scale=0.7]{resources/14.png}\\

\textbf{15.} Bridged mode replicates another node on the physical network and the VM will receive it's own IP address if DHCP is enabled in the network. It can be accessed by all computers in your host network.\\

\textbf{16.} To find the DHCP, use:
\begin{verbatim}
sudo grep -R "DHCPOFFER" /var/log/syslog
\end{verbatim}
\includegraphics[scale=0.6]{resources/15.png}\\
The last line corresponds to the DHCP offer from the server. The IP address of the server is 10.5.192.1.\\

\textbf{17.} To request a new lease from the DHCP server, we use:
\begin{verbatim}
sudo dhclient <interface>
\end{verbatim}
If we apply it to the VM, here is the traffic captured by wireshark:\\~\\
\includegraphics[scale=0.6]{resources/16.png}\\

\end{document}
