\documentclass[conference]{IEEEtran}
\IEEEoverridecommandlockouts
\usepackage{cite}
\usepackage{amsmath,amssymb,amsfonts}
\usepackage{algorithmic}
\usepackage{graphicx}
\usepackage{textcomp}
\usepackage{xcolor}
\def\BibTeX{{\rm B\kern-.05em{\sc i\kern-.025em b}\kern-.08em
    T\kern-.1667em\lower.7ex\hbox{E}\kern-.125emX}}

\begin{document}

\title{IPv4 scarcity, migration from IPv4 to IPv6 and security aspects with IPv6}

\author{
  \IEEEauthorblockN{Alexander Hoffmann}
  \IEEEauthorblockA{\textit{ECE Paris}\\
  Paris, France \\
  alexander.hoffmann@edu.ece.fr}
  \and
  \IEEEauthorblockN{Hugo Fougeres}
  \IEEEauthorblockA{\textit{ECE Paris}\\
  Paris, France \\
  hugo.fougeres@edu.ece.fr}
  \and
  \IEEEauthorblockN{Jeremy Roca}
  \IEEEauthorblockA{\textit{ECE Paris}\\
  Paris, France \\
  jeremy.roca@edu.ece.fr}
}

\maketitle

\section{Introduction}
At the beginning of networking, computer scientists were trying to determine the most efficient means of routing information so that it could be efficiently transmitted throughout the network. It was found that each connected computer maintains its own local address for the information it receives and that both the internetwork's most populous nodes (known as hubs) and the nodes connected to the most rural nodes do the same. Based on this information, and the IP version 4 address system was created. It uses 32 digital bits to represent addresses, generating a theoretical total limit of 4.3 billion addresses.

In May 2016 , International Internet Registration Authority (IARI), the authoritative organization for Internet numbering, declared the global Internet already fragmented. The IPv4 address space now supports only around 36,000 unique identifiers. At the rate of depletion, many, or even most, of the Internet's current home addresses could cease to exist within a decade. It is therefore not unrealistic to expect many existing IP-based applications and services to disappear from the internet at some time in the foreseeable future. It is likely that the first to go will be file sharing, but it is equally possible that other notable Internet-based services, such as online games, forums, instant messaging, and backup, will go in that order. Certain "workarounds" have been proposed to try to keep the web alive, such as re-using one-time addresses, mixing address pairs, and distributed denial of service. But the picture is grim.

Due to degradation in the availability of new IPv4 addresses, some of the community switched to IPv6. New trends in Internet innovation and increased usage of broadband make IPv6 an attractive choice. The most recent public audit, however, found that the Internet's IPv6 address space is still shrinking. In 2008, the Internet reached a global peak of IPv6 addresses.

This paper discusses the depletion of IPv4 addresses. The paper also reviews replacement strategies. Currently, IPv6 is the most widely cited replacement candidate for IPv4. This document analyses IPv6 as an example replacement and goes over technical and security components of IPv6.

\section{IPv4 Address Shortage}

\end{document}
